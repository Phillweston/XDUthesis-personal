%!TEX root = ../Demo.tex
% 中文摘要
\begin{abstract}
在过去的十年中,四旋翼飞行器的受欢迎程度稳步上升。它们被用于军事、消防和救援巡检等领域,并随着公众的普及,逐渐进入民用级、消费级市场。目前以DJI为主的商业公司已具备完善的软硬件体系,但由于其商业性质较浓厚,缺乏对于开源协议和开源地面站的兼容。与此同时,\textbf{Ardupilot}和\textbf{PX4}作为飞控开源界最大的两个项目,采用\textbf{QGroundControl}作为地面站软件并通过\textbf{MAVLink协议}实现空地间的数据传输。因此,市面上迫切需要一种解决方案将开源地面站的相关功能与DJI飞行器MSDK接口进行结合,并通过MAVLink协议开放其飞行控制接口。

\textbf{卡尔曼滤波器(KF)}作为目前较为优秀的滤波和最优估计算法,对于检测后的降落坐标点进行最优估计,最大程度上降低了因误识别或扰动等导致的降落坐标误差,提升了飞行器的降落精准性。传统的基于\textbf{比例、积分、微分(PID)}的控制器存在控制超调、振荡等因素,控制效果随着飞行器的负载和外界扰动的变化而变化。\textbf{模型预测控制(MPC)}做为替代方案,它能够通过最小化参考值和预测输出之间的误差来生成控制系统所需的输入序列,从而根据特定的飞行器的数学模型设计特定的控制模型,从根本上避免了PID的超调、振荡的问题。

为了实现无人机的自主航点飞行巡检以及视觉引导自动降落,并结合开源地面站的功能,本课题利用DJI御2行业进阶版无人机,使用\textbf{DJI MSDK接口},利用无人机端侧遥控器通过VPN以太网数据链路远程组网传输MAVLink协议帧信息至远端的QGC地面站软件。无人机根据地面站给出的航点信息进行航点飞行,航点飞行结束后视觉识别降落点并使用卡尔曼滤波器得到最优估计后的降落点坐标,进一步基于MPC控制引导降落至特定降落目标点。

\end{abstract}
\keywords{MPC控制, DJI MSDK, MAVLink, 目标检测}

% 英文摘要
\begin{enabstract}
Over the past decade, the popularity of quadcopters has steadily increased. They are used in the fields of military, fire and rescue inspection, and gradually enter the civilian and consumer markets with the popularity of the public.

At present, commercial companies based on DJI have a perfect software and hardware system, but due to their strong commercial nature, they lack compatibility with open source protocols and open source ground stations. At the same time, Ardupilot and PX4, as the two largest projects in the open source industry of flight control, use \textbf{QGroundControl} as ground station software and achieve data transmission between air and ground through the \textbf{MAVLink protocol}. Therefore, there is an urgent need for a solution that combines the relevant functions of open source ground stations with the MSDK interface of DJI aircraft and opens up its flight control interface through the MAVLink protocol.
  
\textbf{Kalman filter (KF)} as the current excellent filtering and optimal estimation algorithm, the optimal estimation of the detected landing coordinate point, to the greatest extent to reduce the landing coordinate error caused by misidentification or disturbance, etc., improve the landing accuracy of the aircraft. Traditional \textbf{scale, integration, differentiation (PID)} controllers have control overshoot, oscillation and other factors, and the control effect changes with the load of the aircraft and the change of external disturbances. \textbf{Model Predictive Control (MPC)} as an alternative, it can generate the input sequence required by the control system by minimizing the error between the reference value and the prediction output, so as to design a specific control model according to the mathematical model of a specific aircraft, fundamentally avoiding the problem of overshoot and oscillation of PID.
  
In order to realize the autonomous waypoint flight inspection and visual guidance automatic landing of the UAV, and combined with the functions of the open source ground station, this project uses the DJI Royal 2 Industry Advanced UAV, using \textbf{DJI MSDK interface}, using the drone end-side remote control to remotely network through the VPN Ethernet data link to transmit the MAVLink protocol frame information to the remote QGC ground station software. The UAV conducts a waypoint flight according to the waypoint information given by the ground station, visually identifies the landing point after the waypoint flight and uses the Kalman filter to obtain the optimal estimated landing point coordinates, and further guides the landing to a specific landing target point based on MPC control.

\end{enabstract}
\enkeywords{MPC Control, DJI MSDK, MAVLink, Object Detection}