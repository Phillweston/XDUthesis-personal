%!TEX root = ../Demo.tex
% 中文摘要
\begin{abstract}
  本课题利用DJI御2 行业进阶版无人机,使用DJI MSDK接口,利用无人机端侧遥控器通过以太网数据链路远程组网传输MAVLink协议帧信息至远端的自研地面站软件。无人机根据地面站给出的航点信息自主规划最优航线,航线结束后视觉识别降落点并使用基于MPC控制引导降落至特定降落目标点。
  
  本课题任务为:
  1. 实现视频流压缩与编解码,并基于第三方 VPN 组网技术组建虚拟局域网,通过 RTMP 流视频协议实时传输双光吊舱视频,通过 MAVLink 协议传输无人机相关状态信息和控制指令。
  2. 使用 OpenCV,在无人机的带屏遥控器端基于阈值+边缘检测+霍夫圆检测实现地面端H橙黄色降落点检测,并并通过调用无人机的机载 IMU 和气压计等传感器信息,使用MPC模型预测控制代替传统PID算法完成降落自动控制,从而实现视觉引导自主降落。
\end{abstract}
\keywords{MPC Control, DJI MSDK, MAVLink, 目标检测}

% 英文摘要
\begin{enabstract}
  This project uses the DJI Royal 2 Industry Advanced Edition UAV, uses the DJI MSDK interface, and uses the UAV end-side remote control to remotely network to transmit MAVLink protocol frame information to the remote end of the self-developed ground station software through the Ethernet data link. The UAV independently plans the optimal route according to the waypoint information given by the ground station, visually identifies the landing point after the end of the route and uses MPC-based control to guide the landing to a specific landing target point.
  
  The tasks of this project are:
  1. Realize video stream compression and codec, and form a virtual local area network based on third-party VPN networking technology, transmit dual-optical pod video in real time through RTMP streaming video protocol, and transmit UAVs-related status information and control instructions through MAVLink protocol.
  2. Using OpenCV, the screen remote control of the UAV is based on threshold + edge detection + Hof circle detection to achieve H orange-yellow landing point detection at the ground end, and by calling the UAV's airborne IMU and barometer and other sensor information, the MPC model predictive control is used instead of the traditional PID algorithm to complete the landing automatic control, so as to achieve visual guidance for autonomous landing.
\end{enabstract}
\enkeywords{MPC Control, DJI MSDK, MAVLink, Object Detection}