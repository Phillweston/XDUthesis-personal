%!TEX root = ../Demo.tex
\begin{thanksfor}
大学本科的四年学习即将结束,感谢导师对于项目的指导,感谢实验室的成员的共同参与加快项目的发展。

从大一到大四,从简易的电动直升机到匿名/无名STM32电赛简易四旋翼飞行器再到基于pixhawk的多旋翼飞行器,从大疆NAZA飞控到A3 Pro商业级飞控,从自研二轴倾转飞行器到V22鱼鹰型无人机,从开源VTOL 4+1控制到VTOL倾转矢量控制无人机。我的每一步发展都是基于之前的相关经验和教训,一步步地将飞控算法和导航算法等做到比上一次更好。与此同时,我也在尝试将传统的PID算法进行不断地改良,优化控制模型,采用ADRC自抗扰控制等提高飞行器的控制鲁棒性。

感谢开源界两大教科书式项目Ardupilot和PX4,感谢世界上对于开源飞控维护的社区与团队,使得我们能从中学到很多相关的知识,了解飞控参数调试和算法优化的一些经验。同时,作为Ardupilot官方论坛的成员,我也积极参与论坛内的相关技术讨论,进一步了解当前开源飞控系统的发展趋势、硬件性能、以及相关新功能适配等。

飞控与导航和视觉识别领域作为无人机的小脑、大脑和眼睛,是无人机自主飞行不可缺少的部分,可能因为一个参数错误就会导致飞控系统出现故障,进而导致无人机的姿态不可控甚至炸机。因此,对于无人机领域的开发者来说,更要求的是一种严谨的态度,需要严密的数学计算与仿真以及上机测试从而确保系统不会出故障。

因此,我们在试飞无人机与测试代码之前,也充分考虑到了可能出现的情况,并加以防范,先是飞控软件硬件在环仿真,然后是室内护网飞行,最后是室外实际测试。
\end{thanksfor}